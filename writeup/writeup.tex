\documentclass[a4paper,12pt]{article}

\begin{document}
\title{A Curious Case of Complex Codomain Coloring\thanks{Completely curtailing the less cumbersome "Domain coloring"}}
\author{Jake Looney\\ Undergrad at the University of Tennessee Knoxville}
\date{\today}
\maketitle

\pagenumbering{roman}
\tableofcontents
\newpage
\pagenumbering{arabic}

\section{Motivation}
I ain't gonna lie, I was really high when I wrote the code for this and was trying to just make a domain coloring grapher.
I had seen a coupel of pictures on Wikipedia using domain coloring and saw the term so I decided to just have a go at it.
I was thinking that "domain coloring" would make sense to color the domain and then apply the function.
Of course, normal domain coloring works the other way, where you apply the function then color based on where the points land.
So, what resulted is the other direction of the usual way to do this: points are colored based on their preimage.
This requires some working around in order to work with general functions.
There's not many reasons to do this: it goes against the standard, its computationally intensive, slow as shit, blocky, imprecise, and only shows one point in the preimage per point in the image.
However, despite this, I did discover something pretty interesting while shoving fun functions into it. But first, an explanation of the code.

\section{Code explanation}
There is an attached GitHub repository containing the code. You can view it here:
There are three main functions that are used to create these graphs. The function \verb|createPicture| takes a map of pixel idices and complex points to output a bitmap image. It mainily utiliizes a bitmap implementation I stole from class, but importantly it colors ponits as follows:

This can be changed in order to alter the colorscheme, but I have no idead what the hell LAB color is so I will have to come back to it.\\
The function 



\end{document}
